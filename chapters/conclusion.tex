\chapter{Conclusion}
The aim of this thesis has been an extension of the split-cylinder resonator method by replacing the original model with a new asymmetric model. The model was expected to account for the geometric imperfections of a resonator and by that means improve the accuracy of the method. Out of that reason, we have developed two, new asymmetric models in this thesis, a measurement model and a calibration model. We have explained both models in great detail   and we have studied the convergence of each of them. Using an uncertainty calculation of dielectric measurements carried out with the resonator, we have compared the new asymmetric models to the Janezic model. The comparison has shown that the new asymmetric model could reduce the uncertainty of dielectric constant measurements with a split-cylinder resonator by more than 70\%. We have also implemented the new models in software and used the software to carry out dielectric measurements with a split-cylinder resonator. Using the software we have conducted multiple measurement studies, which included measurements of microwave substrates and plastics, a repeatability evaluation and broadband measurements.

Apart from developing and testing new asymmetric models, we have also shown a new custom measurement setup for the resonator, which we equipped with a novel electro-mechanical coupling adjustment mechanism. We have recognized that this mechanism could be used to make loss tangent measurements more accurate by enforcing symmetric coupling at any time, since symmetric coupling allows us to determine the unloaded quality factor of a mode more accurately. Additionally, we have simplified our measurement setup compared to previous studies by using a self-aligning resonator mount instead of a rigid one. Furthermore, when we touched upon quality factor measurements methods, we have developed an expression for the reflection coefficient of reflection-type measurements and an expression for the transmission coefficient of transmission-type measurements. These expressions explain why the resonance curves are always similarly shaped independent of the coupling network used for the resonator. They might find use in the development of more intricate coupling networks. We have also presented a formula for the transmission coefficient of a split-cylinder resonator, which describes the superposition of the resonances of the transmission coefficient. It has been noted by us that this expression could be used to develop better quality factor measurement methods for the resonator that might be able to resolve multiple overlapping modes.

Although this thesis has discussed many aspects of the split-cylinder resonator method, we hope that the new models and all other results of this thesis will lead to future research in this topic. We recommend that future work should investigate, whether custom split-cylinder resonators for dielectrics could find use in the quality control of microwave substrates instead of less accurate planar circuit methods. We also want to encourage other researcher to develop simpler measurement setups with better clamping mechanism that can apply the same clamping pressure every time a specimen is clamped. We consider this thesis a result of improvements in computer performance that allow more and more complicated models, so we believe that future improvements in computer performance should enable even more accurate models, as long as the dimensional measurement methods are improved at the same time.

