\newgeometry{total={13.5cm, 23.25cm}}
\chapter*{Abstract}
\addcontentsline{toc}{chapter}{Abstract}
The increasing use of microwave signals in consumer electronics has renewed interest in the dielectric properties of materials at microwave frequency bands. The dielectric properties of materials are important parameters in many areas of electronics design, including electromagnetic compatibility, signal integrity, mixed signal circuit, and RF printed-circuit design. Industry and academia therefore require accurate measurement methods for determining the dielectric properties at these frequencies. 

This thesis builds on an existing dielectric measurement method that is widely considered a very important industry standard for dielectric measurements at microwave frequencies, the split-cylinder resonator method, and improves its capabilities and verifies these improvements experimentally. This method determines the dielectric properties of a flat specimen using the TE\st{0np} modes of a resonator, which is formed by the flat specimen and the two halves of a split cylindrical cavity. Previous studies have modelled this resonator as a symmetric resonator with two identical halves. This thesis expands the method by using an asymmetric model instead of the symmetric one. This asymmetric model is supposed to account for geometric imperfections of the two halves, and thereby improve the accuracy of the method. Such geometric imperfections, namely slight asymmetries of the two cavities, are very likely to occur if the two halves of the cylindrical cavity have been made with general purpose tooling. 

This thesis analyses many aspects of the method, including quality factor measurement methods, the coupling of the resonator, and the field configuration of the resonator. Most importantly, it derives a new asymmetric model for the resonator from Janezic's mode-matching model. Since this new asymmetric model cannot be used to calibrate the resonant frequencies and the conductor losses of the resonator, a separate model is derived for that purpose. Furthermore, the convergence of the asymmetric models is studied. These models were also implemented in software to carry out experimental measurements with a custom measurement setup. This setup was equipped with a novel electromechanical coupling adjustment mechanism to facilitate repeatable measurements. Through an uncertainty analysis of dielectric measurements performed with the new model, it is shown that the model can indeed improve the accuracy of dielectric measurements with a split-cylinder resonator. Additionally, the results of four measurement studies with the new model, which include measurements of microwave substrates and plastics, a repeatability evaluation, and measurements with higher modes, are presented.
\pagebreak

\chapter*{Kurzfassung}
\begin{otherlanguage}{german}
Durch die zunehmende Nutzung von Mikrowellen in der Elektronik haben in letzter Zeit die dielektrischen Eigenschaften von Werkstoffen bei diesen Frequenzen immer mehr an Bedeutung gewonnen. Grund hierfür ist die große Bedeutung dieser Eigenschaften für viele Teilgebiete der Elektronik\-entwicklung, wie der Vermeidung von elektromagnetischen Störungen, der Sicherung der Signal\-integrität sowie des Entwurfs von Hoch\-fre\-quenz\-schalt\-ungen und Mixed-\-Signal-\-Schaltungen.

Diese Diplomarbeit beschäftigt sich mit einer dielektrischen Messmethode, der Split-\-cylinder Resonator Methode, die auch ein bedeutender Industriestandard für die Bestimmung der dielektrischen Eigenschaften bei Mikrowellenfrequenzen ist. Diese Arbeit baut hierbei auf früheren wissenschaftlichen Abhandlungen zu dieser Messmethode auf, verbessert sie aber verglichen zu diesen und verifiziert diese Verbesserungen experimentell. Die dielektrischen Eigenschaften von dünnen Dielektrika lassen sich mit dieser Methode unter Verwendungen der TE\st{0np}-Moden eines Resonators bestimmen, der aus dem dünnen Dielektrikum und aus den zwei Hälften eines in der Mitte geteilten kreiszylindrischen Hohlraumresonators gebildet wird. In früheren Abhandlungen wurde dieser Resonator immer mit einem symmetrischen Modell mit zwei identen Hohl\-raum\-resonator\-hälften beschrieben. In dieser Arbeit jedoch wird die Methode erweitert, indem anstatt dieses symmetrischen Modells ein asymmetrisches verwendet wird. Die Aufgabe dieses asymmetrischen Modells ist dabei Größen\-unterschiede zwischen diesen Hohl\-raum\-resonator\-hälften in das Modell einzubeziehen um hierdurch die Genauigkeit der Messmethode zu verhöhen. Solche Größen\-unterschiede treten sehr häufig auf, wenn zur Fertigung der Hohlraum\-resonatoren Standardwerkzeug hergenommen wurde.

Im Rahmen dieser Arbeit werden verschiedene Teilaspekte dieser Messmethode untersucht, unter anderem Methoden zur Bestimmung der Güte, die Kopplung des Resonators und die Feldverteilung im Resonator. In erster Linie wird aber ein neues, asymmetrisches Modell hergeleitet, das von Janezics Mode-Matching Modell abgeleitet wurde. Da ein asymmetrische Modell des gesamten Resonators nicht direkt zu Kalibrierung der Resonanzfrequenzen und der Leiterverluste des Resonators eingesetzt werden kann, wird zusätzlich noch ein Kalibrierungs\-modell hergeleitet. Darüber hinaus wird die Konvergenz beider Modelle überprüft. Für Versuchsmessungen mit einem selbstentwickelten Messaufbau wurden diese Modelle in Software implementiert. Um die Wiederholbarkeit der Messungen mit diesem Messaufbau zu verbessern, wurde der Messaufbau zusätzlich mit einem neuartigen elektro-mechanischen Kopp\-lungs\-anpass\-ungs\-mech\-an\-is\-mus ausgestattet. Mittels einer Fehleranalyse von dielektrischen Messungen mit dem neuen Modell wird gezeigt, dass das Model wirklich die Genauigkeit der dielektrischen Messungen mit dem Resonator erhöhen kann. Zusätzlich zu dieser Fehleranalyse, werden auch die Ergebnisse von vier Versuchsmessungen vorgestellt, in denen verschiedene Mikrowellen\-substrate und Kunststoffe gemessen werden, die Wiederholpräzision der Methode abgeschätzt wird und auch Messungen mit höheren Moden untersucht werden.
\end{otherlanguage}
\pagebreak
\restoregeometry

